
\documentclass[11pt,reqno,a4paper]{amsart}

\usepackage{setspace}
\usepackage[portuguese]{babel}
\usepackage[utf8]{inputenc}
\usepackage{dsfont}
\usepackage{amssymb}
\usepackage{amsmath}
\usepackage{systeme}

\allowdisplaybreaks

\usepackage{fullpage}
\usepackage{setspace}

%newcommands
\newcommand{\Span}[1]{\mathrm{Span}(#1)}
\newcommand{\R}{\mathbb{R}}
\newcommand{\inner}[2]{\langle #1,#2 \rangle}
\newcommand{\norm}[1]{\lVert #1 \rVert}

\def\Assin#1{\noindent\textit{Assinatura}\strut\\%:\\
\framebox[\textwidth]{Gabriel Haruo Hanai Takeuchi\phantom{\vrule height#1}}}

\begin{document}
\parindent=0pt

\title{\textsl{MAT0122 Álgebra Linear I}\\\vspace{3\jot}
  Folha de solução}
\author[MAT0122 Folha de solução]{}

\maketitle
\thispagestyle{empty} 
\pagestyle{plain}
\onehalfspace

\textbf{Nome: Gabriel Haruo Hanai Takeuchi}\hfill
\textbf{Número USP: 13671636}\hspace{3cm}\null

\medskip
\Assin{1cm}

\medskip \textit{Sua assinatura atesta a autenticidade e
  originalidade de seu trabalho e que você se compromete a seguir o
  código de ética da USP em suas atividades acadêmicas, incluindo esta
  atividade.}

\medskip
\textbf{Exercício: E77}\hfill
\textbf{Data: 08/12/2022}\hspace{3cm}\null

\noindent\rule{\textwidth}{0.4pt}

\medskip
\noindent\textbf{SOLUÇÃO}

(i) Let $v_1 = [\alpha_1, \alpha_2, \alpha_3], v_2 = [\beta_1, \beta_2, \beta_3]$.

Then
\begin{align*}
    M =
    \begin{bmatrix}
        \alpha_1 & \beta_1\\
        \alpha_2 & \beta_2\\
        \alpha_3 & \beta_3
    \end{bmatrix}
    M^\intercal =
    \begin{bmatrix}
        \alpha_1 & \alpha_2 & \alpha_3\\
        \beta_1 & \beta_2 & \beta_3
    \end{bmatrix}
\end{align*}
The matrix composition $M^\intercal M$ is
\begin{align*}
    M^\intercal M = 
    \begin{bmatrix}
        \alpha_1 & \alpha_2 & \alpha_3\\
        \beta_1 & \beta_2 & \beta_3
    \end{bmatrix}
    \begin{bmatrix}
        \alpha_1 & \beta_1\\
        \alpha_2 & \beta_2\\
        \alpha_3 & \beta_3
    \end{bmatrix}
    =
    \begin{bmatrix}
        \alpha_1^2 + \alpha_2^2 + \alpha_3^2 & \alpha_1 \beta_1 + \alpha_2 \beta_2 + \alpha_3 \beta_3\\
        \alpha_1 \beta_1 + \alpha_2 \beta_2 + \alpha_3 \beta_3 & \beta_1^2 + \beta_2^2 + \beta_3^2
    \end{bmatrix}
    \\
    =
    \begin{bmatrix}
        \lVert v_1 \rVert^2 & \langle a, b \rangle\\
        \langle a, b \rangle & \lVert v_2 \rVert^2
    \end{bmatrix}
    =
    \begin{bmatrix}
        1 & 0\\
        0 & 1
    \end{bmatrix}
    =
    I_2
\end{align*}

On the other hand, the matrix composition $M M^\intercal$ is
\begin{align*}
    M M^\intercal = 
    \begin{bmatrix}
        \alpha_1 & \beta_1\\
        \alpha_2 & \beta_2\\
        \alpha_3 & \beta_3
    \end{bmatrix}
    \begin{bmatrix}
        \alpha_1 & \alpha_2 & \alpha_3\\
        \beta_1 & \beta_2 & \beta_3
    \end{bmatrix}
    =
    \begin{bmatrix}
        \alpha_1^2 + \beta_1^2 & \alpha_1 \alpha_2 + \beta_1 \beta_2 & \alpha_1 \alpha_3 + \beta_1 \beta_3\\
        \alpha_1 \alpha_2 + \beta_1 \beta_2 & \alpha_2^2 + \beta_2^2 &  \alpha_2 \alpha_3 + \beta_2 \beta_3\\
        \alpha_1 \alpha_3 + \beta_1 \beta_3 & \alpha_2 \alpha_3 + \beta_2 \beta_3 & \alpha_3^2 + \beta_3^2 &\\
    \end{bmatrix}
\end{align*}

It is fairly easy to notice that $M^\intercal M \neq M M^\intercal$ because in this particular case the first one is a two sided square matrix and the other one is a three sided square matrix.

\hrulefill

(ii) The matrix multiplication $M^\intercal M$ has the main diagonal composed of inner products of the same vector
\[
    \inner{v_i}{v_i}, 1 \leq i \leq n
\]
And the other positions composed by inner products of distinct vectors
\[
    \inner{v_i}{v_j}, 1 \leq i \neq j \leq n
\]
With the fact that these vectors are orthonormal in pairs, then
\[\inner{v_i}{v_i} = \norm{v_i}^2 = 1 , \inner{v_i}{v_j} = 0 \]
And therefore,
\begin{align*}
    M^\intercal M =
    \left[
    \begin{array}{ccc}
        v_1 \\
        \hline
        \dots\\
        \hline
        v_n    
    \end{array}
    \right]
    \left[
    \begin{array}{c|c|c}
       v_1 & \dots & v_n 
    \end{array}
    \right]
    =
    \begin{bmatrix}
        \inner{v_1}{v_1} & \inner{v_1}{v_2} & \dots & \inner{v_1}{v_{n-1}} & \inner{v_1}{v_n}\\
        \inner{v_2}{v_1} & \inner{v_2}{v_2} & \dots & \inner{v_2}{v_{n-1}} & \inner{v_2}{v_n}\\
        \vdots\\
        \inner{v_n}{v_1} & \inner{v_n}{v_2} & \dots & \inner{v_n}{v_{n-1}} & \inner{v_n}{v_n}
    \end{bmatrix}
    =
    I_n
\end{align*}

The matrix $M M^T$ is indeed $I_n$ too.
\begin{align*}
    M M^T =
    \left[
    \begin{array}{c|c|c}
        v_1 & \dots & v_n 
    \end{array}
    \right]
    \left[
    \begin{array}{ccc}
        v_1 \\
        \hline
        \dots\\
        \hline
        v_n    
    \end{array}
    \right]
\end{align*}


\endgroup
\end{document}

%%% Local Variables:
%%% mode: latex
%%% eval: (auto-fill-mode t)
%%% eval: (LaTeX-math-mode t)
%%% eval: (flyspell-mode t)
%%% TeX-master: t
%%% End: