#packages
\usepackage{amsmath}


\title{Skip Lists}

\begin{document}
    
\title 

\section*{Definitions}

\begin{itemize}
    \item Estrutura de dados ordenada e probabilística
    \item Generalização das listas ligadas
\end{itemize}    

\section*{Estrutura geral}

Lista ligada convencional (cada célula tem um ponteiro para o próximo elemento).

A diferença é que há células que possuem mais de um ponteiro, com o objetivo de "pular" elementos da lista e fazer uma busca mais rápida.

A quantidade máxima de ponteiros que uma célula tem é pré-definida. Esse número é o número de \textbf{níveis} da lista.

O "nó cabeça" é uma célula que possui $n_{níveis}$ ponteiros para as primeiras células que possuem um ponteiro para cada nível.

Exemplo:
\begin{bmatrix}
    
\end{bmatrix}


\section*{Probabilidades}

Qual a lógica para decidir quais células tem mais ponteiros?

\begin{itemize}
    \item $n$ nós tem pelo menos 1 ponteiro
    \item $n/t$ nós tem pelo menos 2 ponteiros
    \item $n/t^2$ nós tem pelo menos 3 ponteiros
\end{itemize}

\vdots

Generalização: a probabilidade de um nó ter $j+1$ ponteiros é $\frac{1}{j+1}$

Buscas e inserções fazem, em média, $\frac{t \log_t n}{2} = \frac{t}{\log_2 t} \log_2 n$ comparações.

Skip lists têm, em média, $\frac{t}{t-1} n$ ponteiros.

\section*{Busca}

\subsection*{Inserção}
Quais os problemas da inserção?
\begin{itemize}
    \item Como reorganizar os ponteiros?
    \item 
\end{itemize}
\end{abstract}

\subsection*{Remoção}

\section*{Implementação}

\end{document}